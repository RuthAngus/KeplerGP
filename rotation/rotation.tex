\documentclass[12pt,preprint]{aastex}
\usepackage{amsmath}
\usepackage{breqn}
\usepackage{cite,natbib}
\usepackage{epsfig}
\usepackage{cases}
\usepackage[section]{placeins}
\usepackage{graphicx, subfigure}

\begin{document}

\title{Measuring Stellar rotation periods with Gaussian Processes}

\begin{abstract}

None of the commonly used methods for measuring periodic signals are ideally suited to photometric stellar rotation periods.
% Of the most commonly used methods for measuring photometric stellar rotation periods, none is ideally suited to the typically quasi-periodic, non-sinusoidal rotation signals.
% The most commonly used methods for measuring photometric stellar rotation periods are not ideally suited to the problem.
Periodogram and wavelet based methods are not suitable for quasi-periodic, non-sinusoidal signals.
They often cannot distinguish between 'true' periods and harmonics and tend to fail in cases where systematics dominate or substantially contribute to the signal.
The autocorrelation function (ACF) method, (McQuillan et al, 2012) \emph{is} able to distinguish a true period from its harmonics in most cases and is well adapted to quasi-periodic, non-sinusiodal signals.
However, in general it requires heuristics --- hand-tuned, non-motivated-by-science parameters, to determine the `reliability' of a rotation period measurement.
General practise within the field is to use the ACF method, or some combination of ACF, periodogram and wavelets. % is this true?
The major drawback to all of the methods listed above, is that uncertainties on rotation periods are difficult to quantify.
The ideal method for measuring stellar rotation periods would be to use a physically-motivated model and explore the posterior probability distribution of each of its parameters.
Unfortunately, the physics driving the production of variability in stellar light curves is poorly understood and parameters in existing spots models are degenerate.
In particular, spot lifetime and differential rotation (shear) are highly degenerate.
In the absence of a well-motivated physical model, we opt to use a highly flexible, semi-parametric model: a Gaussian process (GP).
The parameters of a GP do not apply to the light curve itself, but to the properties of its covariance structure.
% It is semi-parametric because its parameters do not apply to the light curve itself, but to the properties of the covariance structure of the light curve.
A GP model can cope with quasi-periodic, non-sinusoidal signals, it can model noise and physical signals simultaneously and provides rotation period measurements with well estimated uncertainties.


\end{abstract}

\section{Introduction}

\subsection{Stellar rotation}

Thanks to the high-precision photometry provided by Kepler, stellar rotation periods can be measured directly from light curves.
% Spotted, rotating stars produce variable light curves due to the repeated
The nature of this variability tends to be non-sinusoidal --- star spot patterns are irregular.
Stars often do not rotate as solid bodys, they have differential rotation.
The rotation period of the dominant active region on the stellar surface will show up in the light curve, however this period might not reflect the equatorial rotation period of the star.
Sun spots are born at mid-latitudes and migrate towards the equator, producing the `butterfly diagram' for the sun.
This same effect is seen in stars --- rotation periods do not stay constant, but evolve as if active regions are drifting across latitudes on the surface of a differentially rotating star.
% Stars do not rotate as solid bodys, they have different rotation periods at different latitudes.
% The rotation period of whatever annulus upon which the most dominant spot region lies will show up in the light curve.
% Sun spots are born at mid-latitudes and migrate towards the equator.
% If this is also assumed to happen in solar-type stars, this effect will make the stellar rotation period seem to vary.
% In general it is also quasi-periodic: stars

\subsection{Gaussian processes}

A Gaussian process is any stochastic process in which the joint probability distribution over N samples drawn from the process is Gaussian in N dimensions.
GPs are commonly used in a wide range of scientific fields, such as machine learning, information engineering, biology and cosmology and are now being used more and more in the field of astronomy (Gibson et al, 2011, Aigrain et al, 2012, Foreman-Mackey et al, 2014, PULSAR PEOPLE).
The popularity of GPs is growing within time domain astronomy now that fast GP methods are being developed (cite Dan's paper), as previously, especially densely sampled time series like Kepler data, they have been prohibitively computationally expensive.

\section{Method}

We use a GP to model stellar variability in Kepler light curves.

The kernel function parameterises the covariance matrix.
One of the most commonly used kernel functions is the squared exponential:

\begin{equation}
	k_{ij} = A\exp\left[{\frac{-(x_i-x_j)}{2l^2}}\right]
\end{equation}

which is the simplest kernel you can write down --- it has only two parameters, an amplitude, A and a length scale, $l$.
The length scale parameter controls the rate at which the covariance between data points falls away.
Large $l$ will produce a slowly varying mean function where points are correlated with many surrounding data points.
Small $l$ will produce a rapidly varying mean function, where data points are only correlated with their immediate neighbours.

Kernel functions can be periodic --- a simple periodic function might look like the following:

\begin{equation}
	k_{ij} = A\exp\left[{\frac{-\sin^2\left({\frac{\pi(x_i-x_j)}{P}}\right)}{L^2}}\right],
\end{equation}

where P is the period and L is something like a characteristic length scale in phase-space.
This periodic kernel function parameterises the covariance between points of the same phase: only points with similar phases will be highly correlated.

Kernel functions can be added or multiplied together to produce other valid kernel functions.
To construct our quasi-periodic kernel function we multiply a simple periodic kernel with a squared exponential,

\begin{equation}
	k_{ij} = A\exp\left[{\frac{-(x_i-x_j)}{2l_1^2}}\right]\exp\left[\frac{-\sin^2\left(\frac{\pi(x_i-x_j)}{P}\right)}{l_2^2}\right],
\end{equation}

Given a data set, $\mathbf{x_n} = {x_1, x_2, ..., x_n}$, $\mathbf{y_n} = {y_1, y_2, ..., y_n}$, where in our specific case $x$ is time and $y$ is flux, the log-likelihood function will be as follows:

\begin{equation}
	\ln{\mathcal{L}} = -\frac{1}{2}~\left(\mathbf{y_n}^T~K^{-1}~\mathbf{y_n}~ + ~\log{|K| ~+~ n\log(2\pi)} \right),
\end{equation}

where K is the covariance matrix, constructed from $\mathbf{x_n}$ and the selected covariance kernel.

\subsection{Sampling the Posterior}

\subsection{Dealing with the multi-modality}

Figure \ref{fig:fixed_period} shows the posteriors of the three non-periodic parameters: $l_1$, $l_2$ and $A$, when $P$ is held fixed.
Clearly, these three parameters do not have multi-modal posteriors.
The period parameter, on the other hand, does.
One would expect period to have a multi-modal posterior since period values at harmonics and aliases of the true period will have higher probability than surrounding regions.
This introduces a problem --- it is very difficult to sample from multi-modal posteriors, MCMC walkers tend to get `stuck' in regions of high probability and the chains will not converge.
To get around this problem we could sample over a grid in period, running an MCMC over the other three parameters at each value of period.
Alternatively we could try something like nested sampling.
For the moment we have tried setting the walkers off at evenly spaced positions in period.

Think about how you want to index the names of files and plots, etc!

there is still something not quite right - the mean function doesn't seem to return to zero.... I fear this may cause problems...

\begin{figure}[ht]
\begin{center}
\includegraphics[width=6in, clip=true, trim=0 0 0.5in 0]{/Users/angusr/Python/KeplerGP/rotation/old_figs/success.png}
\caption{Posterior distributions for the three non-periodic parameters, when $P$ is held fixed.}
\label{fig:fixed_period}
\end{center}
\end{figure}

\subsection{Computational tractability}

Computing a GP likelihood requires inverting and calculating the determinant of the covariance matrix which can be expensive:
these operations naively scale like $\mathcal{O}(N^3)$.
% Computing a GP likelihood can be expensive as the determinant and inverse of the covariance matrix must be calculated (although, in reality the inverse is never actually computed).
In practise this covariance matrix can be very large --- for example, one quarter of long cadence Kepler data has around 4000 data points and will produce a 4000x4000 covariance matrix.
For the specific problem of rotation period measurement, the likelihood is likely to be very multi-modal and we therefore opt to sample from the posterior using MCMC.
This means that the determininant and inverse of a very large covariance matrix must be calculated once per likelihood call.
It is easy to see that this problem very quickly becomes intractable.

Luckily there are some tricks we can employ to speed the process up. %thank god!
% Decomposing the matrix into smaller matrices, e.g. using a Cholesky decomposition can speed up the process a little.
In practise, covariance matrices are often sparse and this sparsity can be exploited to speed up these computations.
For the purposes of measuring stellar rotation in Kepler data, covariance matrices will be relatively dense.
Subsampling or binning the data could reduce the dimensionality of the GP, however the way in which this will affect the final result is not obvious.
We use the method for fast determinant calculation described in (Sivaram et al, 2014) in which the matrix is decomposed into smaller matrices, etc.
The determinant calculation then scales like $\mathcal{O}(N\log^2{N})$.
%Circulant method scales as $N\log{N}$.

By hierarchically factoring the matrix into a product of block low-rank updates of the identity matrix, the inversion is reduced to $\mathcal{O}nlog^2n$ and the determinant calculation to $\mathcal{O}nlogn$.
We use george (CITE DAN), a fast Gaussian-process package.

\section{Future work}

Figure out how to join quarters together with a GP?

\section{Current working notes}

We include white noise as a free parameter.
If we don't allow white noise to be inferred from the data and just use the quoted uncertainties on the Kepler data, we find that the GP attemps to explain all the signal with the squared exponential term.
i.e. $l_1$, the parameter controlling the squared exponential fall-off wants to be very small.
When making $l_1$ very large, the squared exp term gets very small and the predictive mean looks very periodic.

Tom's idea - run an optimisation over every period, find the highest likelihood - that should be the global maximum.
Then run an MCMC with that period fixed.
Except you want the marginal likelihood...
So shouldn't I be finding the marginal likelihood right now!?
Well if it works this way then I guess it'll work Tom's way too.

I still haven't quite seen the correct period being recovered - trying doubling the number of data points to see if I can constrain the period better.
How will I search for the longer period signals?
Use a course grid and find the maximum likelihood, then 'hone in' on the maxlike bin.
Bin that up and repeat the process.
How many iterations though?
I guess that will depend on the courseness of the grid.
When the maxlike period is found, do an MCMC.
Probably don't want to fix period, but want to have a prior over it so that you are restricted to the local maximum

Plot periodogram and compare.
Generate periodograms for a bunch of lightcurves.

\end{document}
